\documentclass[a4paper,uplatex,12pt]{jsarticle}
\usepackage[dvipdfmx]{graphicx}
\usepackage{pdfoverlay}
\usepackage{ruby,ulem,fancybox}
\pagestyle{empty}


\pdfoverlaySetPDF{../src/ken1.pdf}

\renewcommand{\baselinestretch}{1.4}

\begin{document}
	\vspace*{-1.6zh}
	\hspace{-5zw}\sout{\phantom{平成}}~~ 2 \phantom{年~} 1 \phantom{月~} 27 \phantom{日~~~~} 2\\[-1.5zh]
	\hspace{-4zw}令和\vspace{-1zh}
    
    \begin{minipage}[t]{0.4\textwidth}
    \vspace*{0.1zh}\hspace*{-5zw}
    EMM
    \end{minipage}
    \begin{minipage}[t]{0.6\textwidth}
    \begin{description}
    	\item[薗田~光太郎] 長崎大学大学院工学研究科
    	\item[\phantom{姓名}] \phantom{所属}
    	\item[\phantom{姓名}] \phantom{所属}
    	\item[\phantom{姓名}] \phantom{所属}
    \end{description}	
    \end{minipage}
    
    %発表題目
    \vspace{5zh}
    音響信号の非現実的ミックスの検出
    
    \vspace{1.5zh}
    \parindent = 6zw
    	
    %登壇者
    \ruby{薗田~光太郎}{{\tiny そのだ~こうたろう}}
    
    %生年月日
    \hspace{1.2zw} 1976\hspace{3.3zw} 5\hspace{3.3zw} 19\hspace{6zw} \sout{\phantom{学生}}~\Ovalbox{\phantom{一般}}
    
    %学歴
    2004年3月 東北大学大学院情報科学研究科博士課程後期三年の課程修了
    
    %学位
    博士(情報科学)
    
    %現在の所属
    長崎大学 大学院 工学研究科
    
    %役職名
    助教
    
    %表彰
    \phantom{表彰}
    
    %主な著書
    \phantom{主な著書}
    
    %研究歴
    \phantom{研究歴}
    		
\end{document}